\documentclass[a4paper,11pt]{article}

\usepackage[english]{babel}
\usepackage[utf8x]{inputenc}
\usepackage[parfill]{parskip}
\usepackage{amsmath}
\usepackage{amssymb}
\usepackage{graphicx}
\usepackage{listings}

\lstset{
  language=Java,
  basicstyle=\footnotesize\ttfamily,
  numbers=left,
  breaklines=true,
  frame=lr,
  captionpos=b,
  showstringspaces=false,
  escapeinside={@*}{*@}
}
\newcounter{counter}

\title{ID1020 Laboration 2}
\author{Emil Tullstedt}

\begin{document}
\maketitle

\newpage

\tableofcontents

\newpage

\section{Series and Combinations}
\label{sec:sac}

\subsection{First identity}

Show the identity
\begin{equation} \label{eq:id1assumption}
\sum_{i=0}^nn = \frac{n(n+1)}{2} for\ n \in \mathbb{N}
\end{equation}

Testing identity for $n = 1$

\begin{equation} \label{eq:id1sum1}
\sum_{i=0}^11 = 0 + 1 = 1
\end{equation}

\begin{equation} \label{eq:id1frac1}
\frac{1(1+1)}{2} = \frac{2}{2} = 1
\end{equation}

As $(\ref{eq:id1frac1}) = (\ref{eq:id1sum1})$ is true (\ref{eq:id1assumption}) holds for $n = 1$.

Writing the identity in the form of a recursive function

\begin{equation}
\sum_{i=0}^kk = 
	\begin{cases}
		S_k = S_{k-1} + k\\
		S_0 = 0
	\end{cases} for\ k \in \mathbb{N}
\end{equation}

Induction assumption: The identity holds for an arbitrary $n-1$.
If true, the following equation can be solved

\begin{equation}
S_n = S_{n-1} + n = \frac{n(n + 1)}{2}
\end{equation}

Solution

\begin{equation}
\begin{split}
S_n = S_{n-1} + n = \frac{n(n-1)}{2} + n = \frac{n(n-1)}{2} + \frac{2n}{2} = \\ = \frac{n(n-1) + 2n}{2} = \frac{n^2-n + 2n}{2} = \frac{n^2 + n}{2} = \frac{n(n + 1)}{2}
\end{split}
\end{equation}

Thus, we can prove that (\ref{eq:id1assumption}) is true for all $n \in \mathbb{N}$.

$$\therefore \sum_{i=0}^nn = \frac{n(n+1)}{2} for\ n \in \mathbb{N}$$

\subsection{Second identity}

$$\binom{n}{2} = \frac{n(n-1)}{2} for\ n \in \mathbb{N}$$

Proof
$$\binom{n}{2} = \frac{n!}{2!(n-2)!} = \frac{n \times (n-1) \times (n-2)!}{2!(n-2)!} = \frac{n(n-1)}{2} \therefore$$

\end{document}