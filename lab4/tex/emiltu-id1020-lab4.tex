\documentclass[a4paper,11pt]{article}

\usepackage[english]{babel}
\usepackage[utf8x]{inputenc}
\usepackage[parfill]{parskip}
\usepackage{amsmath}
\usepackage{amssymb}
\usepackage{graphicx}
\usepackage{listings}
\usepackage{hyperref}
\usepackage{algorithmicx}
\usepackage{enumitem}

\lstset{
  language=Java,
  basicstyle=\footnotesize\ttfamily,
  numbers=left,
  breaklines=true,
  frame=lr,
  captionpos=b,
  showstringspaces=false,
  escapeinside={@*}{*@}
}
\newcounter{counter}

\title{ID1020 Laboration 3}
\author{Emil Tullstedt}

\hyphenation{definition}

\begin{document}
\maketitle

\newpage

\tableofcontents

\newpage

\section{Trees}
\label{sec:trees}

\subsection{Graphs \textit{\&} Trees}

Following is an analysis of five directed graphs to determine if they fit into the definition of a tree\footnote{Doesn't contain any cycles \textit{\&} has one element from which every other element can be derived} or, if it isn't a tree, which cycles are contained within the graph.

\begin{enumerate}[label=\bfseries Graph \arabic*:]
\item $G_1$ is a tree with the root in 1
\item $G_2$ is a tree with the root in 1
\item $G_3$ has a cycle 6--2--1
\item $G_4$ is a tree with the root in 6
\item $G_5$ has a cycle 6--2--1--4
\end{enumerate}
\newpage
\subsection{Tree Traversal}

The tuples given below is five traversals of a tree which has it's root 4 which has the children 2 and 6. The children of 2 is 1 and 3 and the children of 6 is 5 and 7.

For the first four traversals, the instructions \textsc{Traverse}($l$), \textsc{Traverse}($r$) and \textsc{Visit}($v$) are utilized. These are parsed as if they were defined

\begin{description}[style=multiline,leftmargin=3cm]
\item[\textsc{Traverse}($l$)] If possible, move to the first child and execute an iteration of the algorithm on that child
\item[\textsc{Traverse}($r$)] Same as \textsc{Traverse}($l$) but for the second child
\item[\textsc{Visit}($v$)] Append the tuple with the current vector $v$
\end{description}

The final traversal (breadth-first) is defined in pseudocode using the definition of \textsc{Visit}($v$) given above. The breadth-first algorithm is parsed as if it begins with the left-most item on a row, appends it to the tuple and then continues from left to right on that row until it finds the right-most item, when it continues in the same manner on the next row.

The list below describes the different scenarios.

\begin{description}[style=multiline,leftmargin=6cm]
\item[Pre-order] \texttt{(4, 2, 1, 3, 6, 5, 7)}
\item[In-order (Ascending)] \texttt{(1, 2, 3, 4, 5, 6, 7)}
\item[In-order (Descending)] \texttt{(7, 6, 5, 4, 3, 2, 1)}
\item[Post-order] \texttt{(1, 3, 2, 5, 7, 6, 4)}
\item[Breadth-first] \texttt{(4, 2, 6, 1, 3, 5, 7)}
\end{description}
\newpage
\subsection{TreeSort}

\end{document}