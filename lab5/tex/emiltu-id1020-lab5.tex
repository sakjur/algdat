\documentclass[a4paper,11pt,notitlepage]{article}

\usepackage[english]{babel}
\usepackage[utf8x]{inputenc}
\usepackage[parfill]{parskip}
\usepackage{amsmath}
\usepackage{amssymb}
\usepackage{graphicx}
\usepackage{listings}
\usepackage[colorlinks=true]{hyperref}
\usepackage{algorithm}
\usepackage{algorithmicx}
\usepackage{algpseudocode}
\usepackage{enumitem}

\lstset{
  language=Java,
  basicstyle=\footnotesize\ttfamily,
  numbers=left,
  breaklines=true,
  frame=lr,
  captionpos=b,
  showstringspaces=false,
  escapeinside={@*}{*@}
}
\newcounter{counter}

\title{ID1020 Laboration 5}
\author{Emil Tullstedt}

\hyphenation{definition algo-rithm}

\begin{document}
\maketitle

\tableofcontents

\section{Graphs}

Because of the definition of a loop, any element within a loop may reach another element within the same loop and therefore any element that may be reached from any element within the loop may be reached from any other element within the loop as well.\footnote{Loopedylooploop}

This can be explained as $^+(v) = ^+(v')$ where both $v$ and $v'$ is in a loop $\hat{G}$ is true because as you can step through the loop until satisfied that the next vertex is any $^+(v)$ for a $(v')$ within the loop $\hat{G}$.

\newpage
\subsection{$G_1$}
\begin{itemize}
\item $G_1 \backslash\sim = \{1, 2, 3, 4, 5, 6, 7\}$
\item $^+(1) = ^+(2) = ^+(3) = ^+(4) = ^+(5) = ^+(6) = ^+(7) = V$
\item $\_(1, 2) = w(1, 2) = \{(1, 3, 2), (3, 5, 1), (5, 7, 1), (7, 6, 2), (6, 2, 3)\}$ of length $5$ and weight $9$
\item $\_(1, 3) = w(1, 3) = \{(1,3,2)\}$ of length $1$ and weight $2$
\item $\_(1, 4) = w(1, 4) = \{(1, 3, 2), (3, 5, 1), (5, 7, 1), (7, 6, 2), (6, 4, 2)\}$ of length $5$ and weight $8$
\item $\_(1, 5) = w(1, 5) = \{(1, 3, 2), (3, 5, 1)\}$ of length $2$ and weight $3$
\item $\_(1, 6) = w(1, 6) = \{(1, 3, 2), (3, 5, 1), (5, 7, 1), (7, 6, 2)\} $ of length $4$ and weight $6$
\item $\_(1, 7) = w(1, 7) = \{(1, 3, 2), (3, 5, 1), (5, 7, 1)\}$ of length $3$ and weight $4$
\end{itemize}

\subsection{$G_2$}
\begin{itemize}
\item $G_2 \backslash\sim = \{\{1, 2, 4, 6\}, \{3\}, \{5\}, \{7\}\}$
\item $^+(1) = ^+(2) = ^+(4) = ^+(6) = V$
\item $^+(3) = \{3, 5, 7\}$
\item $^+(5) = \{5, 7\}$
\item $^+(7) = \{7\}$
\item $\_(1, 2) = w(1, 2) = \{(1, 4, 2), (4, 6, 2), (6, 2, 1)\}$ of length $3$ and weight $5$
\item $\_(1, 3) = w(1, 3) = \{(1, 3, 1)\}$ of length $1$ and weight $1$
\item $\_(1, 4) = w(1, 4) = \{(1, 4, 2)\}$ of length $1$ and weight $2$
\item $\_(1, 5) = \{(1, 3, 1), (3, 5, 10)\}$ of length $2$ and weight $11$
\item $w(1, 5) = \{(1, 4, 2), (4, 6, 2), (6, 2, 1), (2, 5, 2)\}$ of length $4$ and weight $7$
\item $\_(1, 6) = w(1, 6) = \{(1, 4, 2), (4, 6, 2)\}$ of length $2$ and weight $4$
\item $\_(1, 7) = \{(1, 3, 1), (3, 5, 10), (5, 7, 2)\}$ of length $3$ and weight $13$
\item $\_(1, 7) = w(1, 7) = \{(1, 4, 2), (4, 6, 2), (6, 7, 1)\}$ of length $3$ and weight $5$
\end{itemize}

\end{document}